\documentclass[11pt]{report}

\usepackage[utf8]{inputenc}
\usepackage[french]{babel}
\usepackage{fullpage}
\usepackage{amsmath}
\usepackage{microtype}
\usepackage{titlesec}

\title{Rapport d'utilisation de la cellule du Club*Nix}
\date{18 Novembre 2015}

\titleformat{\chapter}{\normalfont\huge\bf}{\thechapter.}{20pt}{\huge\bf}

\begin{document}

\maketitle

\paragraph{} Ce rapport fait suite à la demande du BDE sur l'utilisation de la
cellule du Club*Nix pour le 18 Novembre 2015.

\tableofcontents

\part{Activités du club}

\chapter{Post-assistance}

% This is an intro:
\paragraph{} Pour commencer, un des buts principaux du Club*Nix est
d'accueillir tout élèves à la recherche d'aide en programmation (Java, C, HTML,
\ldots) et d'installation de Linux ou de différents logiciels. Avoir une cellule
permet aux étudiants de savoir où aller pour demander de l'aide.

\paragraph{} Au sein de sa cellule, le Club*Nix organise de nombreuses sessions
de post-a auprès des premières années lors de l'unité de java ou de l'élective
HTML mais également tout étudiant qui aurait des questions sur l'informatique
ou même dans d'autres domaines liés à l'ESIEE.  

\paragraph{} La cellule du Club*Nix nous permet également de préparer divers
tutos tout au long de l'année tels que des tutos Java (pour les premières
années majoritairement, mais des années supérieures y participent également),
des tutos C++ ou des tutos Git pour la gestion de projet. Tous ces tutos se
suivent régulièrement de post-a pour les étudiants qui n'auraient pas pu être
présents lors du tuto. Lors de certains tutos comme le Tuto Linux qui a eu un
gros succès, il a fallu que le club puisse stocker les polys distribués pendant
cet évènement qui ont été demandés plusieurs semaines après le tuto.

\paragraph{} Le Club*Nix est un des seuls club de l'école à proposer de l'aide
aux étudiants dans divers domaines tels que les cours dispensés à l'ESIEE et
les projets personnels. La cellule du Club*Nix est donc utile pour que les
étudiants sachent où trouver l'aide que nous fournissons actuellement.

% TODO: add image of Java tutoring in res/ directory

\chapter{Membres}

% This is also an intro:
\paragraph{} Le club est également un lieu de vie pour la plupart des membres
actifs du club, sur les 56 membres inscrits, au moins 15--20 personnes sont
présents en permanence dans la cellule que ce soit pour des projets perso,
associés à l'ESIEE ou internes au Club*Nix.

\part{Utilisation de la cellule}

\chapter{Inventaire}

% Again, this is an intro:
\paragraph{} De plus, la cellule du Club*Nix nous permet d'avoir 6 ordinateurs
sous Linux à disposition pour tout ESIEEens avec une assistance des membres du
club qui se font un plaisir de partager leurs connaissances.

% This however, is an intro:
\paragraph{} Avoir une cellule nous permet également de posséder un frigo que
l'on remplit régulièrement de canettes ce qui nous permet de faire un peu
d'argent pour obtenir de nouvelles machines qui serviront, comme énoncé
précédemment, à aider les ESIEEens qui désirent développer leurs compétences
informatiques.

\end{document}
