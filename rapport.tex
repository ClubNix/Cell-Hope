\documentclass[11pt]{article}

\usepackage[utf8]{inputenc}
\usepackage[french]{babel}
\usepackage{fullpage}
\usepackage{amsmath}
\usepackage{microtype}

\title{Rapport d'utilisation de la cellule du Club*Nix}
\date{18 Novembre 2015}

\begin{document}

\maketitle

\paragraph{} Ce rapport fait suite à la demande du BDE sur l'utilisation de la
cellule du Club*Nix.

\paragraph{} Pour commencer, un des buts principaux du Club*Nix est
d'accueillir tout élèves à la recherche d'aide en programmation (Java, C, HTML,
\ldots) et d'installation de Linux ou de différents logiciels. Avoir une cellule
permet aux étudiants de savoir où aller pour demander de l'aide.

\paragraph{} Le club est également un lieu de vie pour la plupart des membres
actifs du club, sur les 56 membres inscrits, au moins 15--20 personnes sont
présents en permanence dans la cellule que ce soit pour des projets perso,
associés à l'ESIEE ou internes au Club*Nix.


\paragraph{} De plus, la cellule du Club*Nix nous permet d'avoir 6 ordinateurs
sous Linux à disposition pour tout ESIEEens avec une assistance des membres du
club qui se font un plaisir de partager leurs connaissances.

\paragraph{} Avoir une cellule nous permet également de posséder un frigo que
l'on remplit régulièrement de canettes ce qui nous permet de faire un peu
d'argent pour obtenir de nouvelles machines qui serviront, comme énoncé
précédemment, à aider les ESIEEens qui désirent développer leurs compétences
informatiques.

\end{document}
