\documentclass[11pt]{report}

\usepackage[utf8]{inputenc}
\usepackage[french]{babel}
\usepackage[hidelinks]{hyperref}
\usepackage{fullpage}
\usepackage{amsmath}
\usepackage{microtype}
\usepackage{titlesec}

\title{Rapport d'utilisation de la cellule du Club*Nix}
\date{18 Novembre 2015}

\titleformat{\chapter}{\normalfont\huge\bf}{\thechapter.}{20pt}{\huge\bf}

\begin{document}

\maketitle

\paragraph{} Ce rapport fait suite à la demande du BDE sur l'utilisation de la
cellule du Club*Nix pour le 18 Novembre 2015.

\tableofcontents

\part{Activités du club}

\chapter{Post-assistance}

% This is an intro:
\paragraph{} Pour commencer, un des buts principaux du Club*Nix est
d'accueillir tout élèves à la recherche d'aide en programmation (Java, C, HTML,
\ldots) et d'installation de Linux ou de différents logiciels. Avoir une cellule
permet aux étudiants de savoir où aller pour demander de l'aide.

% TODO: add image of Java tutoring in res/ directory

\chapter{Membres}

% This is also an intro:
\paragraph{} Le club est également un lieu de vie pour la plupart des membres
actifs du club, sur les 56 membres inscrits, au moins 15--20 personnes sont
présents en permanence dans la cellule que ce soit pour des projets personnels,
associés à l'ESIEE ou internes au Club*Nix.

\part{Utilisation de la cellule}

\chapter{Inventaire}

\section{Machines}

% Again, this is an intro:
\paragraph{} La cellule du Club*Nix nous permet d'avoir 6 ordinateurs sous
Linux, chacun ayant une distribution différente, à disposition pour tout
ESIEEens avec une assistance des membres du club qui se font un plaisir de
partager leurs connaissances.

\paragraph{} Nous avons aussi à l'intérieur de la cellule 3 serveurs nous
permettant de desservir le site tu Club *Nix situé à l'adresse
\url{https://clubnix.fr}. Ils nous permettent aussi de stocker les fichiers et
dossiers des membres, ainsi que de leur fournir un compte complet avec
identifiant et mot de passe.

\section{Matériel à disposition}

% This however, is an intro:
\paragraph{} Avoir une cellule nous permet également de posséder un frigo que
l'on remplit régulièrement de canettes ce qui nous permet de faire un peu
d'argent pour obtenir de nouvelles machines qui serviront, comme énoncé
précédemment, à aider les ESIEEens qui désirent développer leurs compétences
informatiques.

\paragraph{} Le club mets aussi à disposition des snacks comme des barres de
Mars, des Kit Kats, etc\ldots, toujours à disposition des membres, ainsi qu'une
bouilloire et une cafetière et de la vaisselle pour permettre aux membres de
démarrer la journée de manière plus productive.

\paragraph{} Nous mettons aussi à disposition du matériel de stockage comme des
casiers réservés à des membres, ce qui permet par exemple, de venir à l'école
avec un sac moins pesant, ou d'avoir des documents à portée de main lorsque le
membre est dans la cellule.

\paragraph{} Avec cela sont proposés aussi des outils mécaniques comme des
tournevis, pinces, cutters, mais aussi des outils électroniques comme quelques
composants, un fer à souder, et du matériel informatique comme des claviers,
souris, câbles Ethernet, câbles d'alimentation, USB, micro-USB, multiprises,
enceintes, switch, etc\ldots

\paragraph{} Du matériel scolaire et aussi disponible pour les membres comme
des dictionnaires d'anglais, des calculatrices, des règles et d'autres
instruments de mesures.

\paragraph{} Il est aussi possible de lire dans la cellules des comics et
bandes-dessinées, des mangas, des livres/références/guides sur des sujet
principalement dans le domaine de l'informatique, comme le réseau, la sécurité,
la programmation.

\chapter{Membres}

\paragraph{} Comme mentionné précédemment de la première partie, les membres du
Club *Nix viennent régulièrement dans la cellule et l'utilisent à fin de
travailler sur diverse matières, utilisent les ordinateurs du club pour
développer, apprendre, ou même aller sur internet ou regarder leurs mails.

\paragraph{} Ils utilisent la cellule également afin de stocker leurs projets
scolaires ou non, et travaillent aussi dessus à l'intérieur de la cellule.

\paragraph{} Enfin et surtout, la cellule est utilisée comme lieu de vie pour
échanger et communiquer entre les membres, ce qui en fait pour les membres un
lieu privilégié pour travailler et discuter de sujets plus large que dans le
domaine de l'informatique.

\end{document}

% vim: spell : spelllang=fr
